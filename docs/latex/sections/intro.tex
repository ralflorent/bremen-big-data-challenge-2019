% Bremen Big Data Challenge - Edition 2019
%
% Data Analysis Competition
%
% Team `gentleman`
%
% Created on March 10, 2019
%
% Authors:
%   Gari Ciodaro <g.ciodaroguerra@jacobs-university.de>
%   Diogo Cosin <d.ayresdeoliveira@jacobs-university.de>
%   Ralph Florent <r.florent@jacobs-university.de>
%
% Introduction block for the documentation

% ==============================================================================
% START: Introduction
% ==============================================================================

\section{Introduction}

\textit{"The Bremen Big Data Challenge"} is a competition organized on a yealy
basis by the Universität Bremen in Bremen, Germany. This is a challenge in which 
a data set is provided and a Big Data task, specified. In this year's
edition, teams were supposed to come up with a solution to a classification 
task. According to the data collected by different sensors positioned in the leg
of human subjects, teams should classifiy different leg movements. The team with
the best \textit{Accuracy} is then consecrated the competition winner.

This report describes the data workflow implemented to solve the task proposed
in \textit{"The Bremen Big Data Challenge"}. We begin by descring the data
set structure. Posteriorly, a brief theoretical review is introduced for the 
concepts involved in the workflow. The data pipeline is then presented. The
strategies adopted in the Data Preprocessing as well as the Data Exploitation
are presented. Finally, the results obtained are revealed and discussed.


% ==============================================================================
% START: Introduction
% ==============================================================================