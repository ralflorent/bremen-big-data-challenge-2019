% Bremen Big Data Challenge - Edition 2019
%
% Data Analysis Competition
%
% Team `gentleman`
%
% Created on March 10, 2019
%
% Authors:
%   Gari Ciodaro <g.ciodaroguerra@jacobs-university.de>
%   Diogo Cosin <d.ayresdeoliveira@jacobs-university.de>
%   Ralph Florent <r.florent@jacobs-university.de>
%
% Contents block for the documentation
%
% 1) Introduction
% 2) Background of the data (data description)
% 3) Data preprocessing
% 4) Data exploitation
% 5) Data Analysis
% 6) Conclusion / Results
% 7) References / Bibliography

% ==============================================================================
% START: Methods, Results, Discussions, Conclusion
% ==============================================================================

\section{Background of the Data}
\subsection{Data Source}

Provided by \emph{"The Bremen Big Data Challenge 2019" Organizers}, the collected data are based on
daily athletic movements \parencite{bbdc}. Using wearable sensors above and below the knee of the individual (athletic),
a dataset of 19 individuals, mainly identified as \emph{subjects}, has been recorded. And as the
competition requires, the data of 15 subjects are used as the training dataset and the remaining data
as the testing dataset. The dataset is publicly available online on the official website:
\href{https://bbdc.csl.uni-bremen.de/images/2019/bbdc_2019_Bewegungsdaten_mit_referenz.zip}{BBDC}.

\subsection{Description and Format}

The data comprise the following 22 movements:
\begin{itemize}
    \item Race ('run')
    \item Walking ('walk')
    \item Standing (standing)
    \item Sitting ('sit')
    \item Get up and sit down ('sit-to-stand', 'stand-to-sit')
    \item Up and down stairs ('stair-up', 'stair-down')
    \item Jump on one or both legs ('jump-one-leg', 'jump-two-leg')
    \item Run left or right ('curve-left-step', 'curve-right-step')
    \item Turn left or right on the spot, left or right foot first ('curve-left-spin-Lfirst',
    'curve-left-spin-Rfirst', 'curve-right-spin-Lfirst', 'curve-right- spin-Rfirst ')
    \item Lateral steps to the left or right ('lateral-shuffle-left', 'lateral-shuffle-right')
    \item Change of direction when running to the right or left, left or right foot first
    ('v-cut-left-left', 'v-cut-left-right', 'v-cut-right-left', 'v-cut' right-Rfirst ')
\end{itemize}

All the data are available as CSV files, or Comma-Separated Values. Starting from the training
dataset identified by the \textbf{\emph{train.csv}}" file, the dataset is fixed and formatted  as
described in the following lines:





\section{Data Preprocessing}

\section{Data Exploitation}

\section{Data Analysis}

\section{Results and Discussions}

\section{Conclusion}

% ==============================================================================
% START: Methods, Results, Discussions, Conclusion
% ==============================================================================